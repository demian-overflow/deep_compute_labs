\documentclass{article}
\usepackage{amsmath, amssymb, amsthm}
\usepackage{hyperref}
\usepackage{enumitem}
\usepackage{graphicx}
\usepackage{booktabs}
\usepackage{geometry}

\geometry{margin=1in}

\title{Mathematical Foundations for Deep CS, ML, and Cryptography}

\begin{document}

\maketitle

\section{Overview}

This document outlines a comprehensive learning plan covering the mathematical foundations needed to understand machine learning research papers, cryptography (including blockchain applications), and computer science fundamentals.

\section{Core Mathematical Foundations}

\subsection{Phase 1: Essential Mathematics}

\subsubsection{Linear Algebra}
\begin{itemize}[leftmargin=*]
    \item \textbf{Topics:} Vectors, matrices, linear transformations, eigenvalues/eigenvectors, vector spaces
    \item \textbf{Applications:} Neural networks, dimensionality reduction, computer graphics
    \item \textbf{Resources:}
    \begin{itemize}
        \item ``Linear Algebra Done Right'' by Sheldon Axler
        \item Gilbert Strang's MIT OCW Linear Algebra course
        \item 3Blue1Brown's ``Essence of Linear Algebra'' video series
    \end{itemize}
\end{itemize}

\subsubsection{Calculus}
\begin{itemize}[leftmargin=*]
    \item \textbf{Topics:} Limits, derivatives, integrals, multivariable calculus, partial derivatives
    \item \textbf{Applications:} Optimization in ML, gradient descent, backpropagation
    \item \textbf{Resources:}
    \begin{itemize}
        \item ``Calculus'' by James Stewart
        \item MIT OCW Calculus courses
        \item 3Blue1Brown's ``Essence of Calculus'' series
    \end{itemize}
\end{itemize}

\subsubsection{Probability \& Statistics}
\begin{itemize}[leftmargin=*]
    \item \textbf{Topics:} Random variables, distributions, expectation, hypothesis testing, Bayesian statistics
    \item \textbf{Applications:} Statistical ML models, uncertainty quantification, cryptographic security
    \item \textbf{Resources:}
    \begin{itemize}
        \item ``All of Statistics'' by Larry Wasserman
        \item ``An Introduction to Statistical Learning'' by James, Witten, Hastie, Tibshirani
    \end{itemize}
\end{itemize}

\subsubsection{Discrete Mathematics}
\begin{itemize}[leftmargin=*]
    \item \textbf{Topics:} Logic, set theory, combinatorics, recursion, graph theory
    \item \textbf{Applications:} Algorithms, data structures, automata theory
    \item \textbf{Resources:}
    \begin{itemize}
        \item ``Discrete Mathematics and Its Applications'' by Kenneth Rosen
        \item ``Concrete Mathematics'' by Graham, Knuth, and Patashnik
    \end{itemize}
\end{itemize}

\subsection{Phase 2: Advanced Topics}

\subsubsection{Abstract Algebra}
\begin{itemize}[leftmargin=*]
    \item \textbf{Topics:} Groups, rings, fields, Galois theory
    \item \textbf{Applications:} Cryptography, error-correcting codes, quantum computing
    \item \textbf{Resources:}
    \begin{itemize}
        \item ``Abstract Algebra'' by Dummit and Foote
        \item ``A Book of Abstract Algebra'' by Charles Pinter
    \end{itemize}
\end{itemize}

\subsubsection{Number Theory}
\begin{itemize}[leftmargin=*]
    \item \textbf{Topics:} Modular arithmetic, prime numbers, congruences, primality testing
    \item \textbf{Applications:} Public-key cryptography, blockchain, hashing
    \item \textbf{Resources:}
    \begin{itemize}
        \item ``Elementary Number Theory'' by David Burton
        \item ``An Introduction to the Theory of Numbers'' by Hardy and Wright
    \end{itemize}
\end{itemize}

\subsubsection{Information Theory}
\begin{itemize}[leftmargin=*]
    \item \textbf{Topics:} Entropy, mutual information, channel capacity, coding theory
    \item \textbf{Applications:} Compression, ML, cryptography
    \item \textbf{Resources:}
    \begin{itemize}
        \item ``Elements of Information Theory'' by Cover and Thomas
        \item ``Information Theory, Inference, and Learning Algorithms'' by David MacKay
    \end{itemize}
\end{itemize}

\subsubsection{Optimization Theory}
\begin{itemize}[leftmargin=*]
    \item \textbf{Topics:} Convex optimization, gradient methods, Lagrangian duality
    \item \textbf{Applications:} Training ML models, operations research
    \item \textbf{Resources:}
    \begin{itemize}
        \item ``Convex Optimization'' by Boyd and Vandenberghe
        \item ``Numerical Optimization'' by Nocedal and Wright
    \end{itemize}
\end{itemize}

\section{Specialized Tracks}

\subsection{Machine Learning Mathematics}
\begin{itemize}[leftmargin=*]
    \item Functional analysis (Hilbert spaces, kernels)
    \item Computational linear algebra
    \item Probabilistic graphical models
    \item Differential geometry (for manifold learning)
    \item Statistical learning theory
\end{itemize}

\subsection{Cryptography \& Blockchain Mathematics}
\begin{itemize}[leftmargin=*]
    \item Elliptic curve cryptography
    \item Zero-knowledge proofs
    \item Hash functions and collision resistance
    \item Merkle trees and verification
    \item Secure multi-party computation
\end{itemize}

\subsection{Theoretical Computer Science}
\begin{itemize}[leftmargin=*]
    \item Computability theory
    \item Algorithmic complexity
    \item Formal languages and automata
    \item Type theory
    \item Category theory
\end{itemize}

\section{Learning Approach}

\begin{enumerate}
    \item \textbf{Foundations First:} Master core topics before specializing
    \item \textbf{Practice Problems:} Implement concepts in code using languages in this repo
    \item \textbf{Applied Projects:} Create mini-projects demonstrating each concept
    \item \textbf{Paper Reading:} Start with seminal papers, gradually moving to recent research
    \item \textbf{Interdisciplinary Connections:} Explore how concepts connect across domains
\end{enumerate}

\section{Suggested Learning Projects}

\begin{enumerate}
    \item Implement basic cryptographic primitives (RSA, elliptic curves)
    \item Build a simple neural network from scratch using only linear algebra
    \item Create visualization tools for mathematical concepts
    \item Implement common ML algorithms using only numpy
    \item Develop simple blockchain primitives focusing on cryptographic aspects
\end{enumerate}

\section{Assessment Strategies}

\begin{itemize}[leftmargin=*]
    \item Implement mathematical algorithms in C, Rust, and CUDA
    \item Reproduce results from foundational papers
    \item Create teaching notebooks explaining complex concepts
    \item Contribute to open-source math/ML/crypto libraries
\end{itemize}

\vspace{1cm}
\noindent This plan follows a progressive approach, building fundamentals before tackling advanced topics. Adjust the pace based on your background and specific interests in ML, cryptography, or theoretical CS.

\end{document} 